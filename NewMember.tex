\documentclass[10pt,letterpaper]{article}
\usepackage{fullpage}
\usepackage[utf8x]{inputenc}
\usepackage{ucs}
\usepackage{fixltx2e}
\usepackage{graphicx}
\usepackage{pdfpages}
\usepackage{longtable}
\usepackage[colorlinks=true,urlcolor=blue]{hyperref}
\hypersetup{ 
      pdfauthor   = {Andrew G. Meyer}, 
      pdftitle    = {i3Detroit New Member Form} 
} 
\begin{document}
    \author{Andrew G. Meyer}
    \title{i3Detroit New Member Form}
    \date{\today}
    
    \section*{\centering Welcome to i3Detroit!}
\noindent
As a community-run organization, we place a lot of trust in each other, and we all help keep things running around here. There's a lot to learn, and this page is meant as a guide. Hang on to it.

\noindent
Here are some very knowledgeable members who would love to help you get started. Call or email any of us with questions. We were all new, once...

    \begin{enumerate}
        \item Nate Bezanson, 248-379-3614, \href{mailto:myself@telcodata.us}{myself@telcodata.us}
        \item Maddy Winans, 248-821-0798, \href{mailto:madalyn.winans@gmail.com}{madalyn.winans@gmail.com}
        \item Matt Oehrlein, 651-261-9747, \href{mailto:matt@i3detroit.com}{matt@i3detroit.com}
        \item Your mentor, see other side of this page...
    \end{enumerate}

\noindent
As a new member:
    \begin{itemize}
        \item Please accept all the Google Groups invitations in your email -- this is our primary communication method. If you didn't get them, email \href{mailto:contact@i3detroit.com}{contact@i3detroit.com}! (Check your spam box...)
        \item Go to \url{http://groups.google.com/group/i3detroit} and post a ``Hi I'm new'' message. If you're unable to log in, get in touch with your mentor, or email \href{mailto:contact@i3detroit.com}{contact@i3detroit.com}.
        \item If you don't know how to use a tool, feel free to post to the group above, asking to learn. Someone will be happy to set up a time to teach you, but you have to ask! Be bold!
        \item There's a box of T-shirts under the front counter. Want one? Slip ten bucks in an envelope, write ``T-shirt'' on it, and slip it into the dues drop-box. (The cup in the fridge can usually make change.)
        \item There's a tremendous amount of information collected in the Wiki. Find it at \url{http://www.i3detroit.com/wiki}. Try searching for the name of any tool, for instance.
        \item If you see something that needs fixing, bring it up and help fix it. If you see someone doing something unsafe, speak up and help them do it safely. If you need help, ask for it.
        \item You have the same rights and responsibilities as every other member, new or ``seasoned''.
            \begin{itemize}
                \item Some of those rights:
                \begin{itemize}
                    \item Give tours, interviews, and stuff.
                    \item Use every tool you've been checked out on. Check with the owner, listed on the wiki.
                    \item Get involved at meetings and via email, and help shape the community.
                    \item Host guests, hold events, and share your own interests with the group.
                    \item Have your projects featured on the i3detroit.com front page blog. Ask how!
                \end{itemize}
                \item Some of those responsibilities:
                \begin{itemize}
                    \item Take out the trash when it's full. You're now one of dozens of part-time janitors!
                    \item Uphold the group's spirit and image, especially on \href{mailto:i3detroit-public@googlegroups.com}{i3detroit-public@googlegroups.com}.
                    \item Help your fellow members keep after our various messes.
                    \item Make sure every guest signs a waiver and gets signed in by a member, perhaps you.
                    \item If you just took the last copy of this page, find the original on the wiki and print more.
                \end{itemize}
            \end{itemize}
        \item When talking about i3Detroit, remember to say ``we''. You're one of us now, welcome!
    \end{itemize}

\newpage

    \section*{\centering i3Detroit Keyholder Responsibilities}
    \centering{(Treat this as a learn-the-community scavenger hunt)}

\small
    \begin{Form}
        \begin{enumerate}
            \item Every member may host \TextField[name=11,width=2.5cm,bordercolor=black,charsize=10pt]{\mbox{}} in the space.
            \item Guests must complete a \TextField[name=21,width=2.5cm,bordercolor=black,charsize=10pt]{\mbox{}} on their first visit.
            \item Guests must sign in on the \TextField[name=31,width=2.5cm,bordercolor=black,charsize=10pt]{\mbox{}} every time they visit.
            \item Members should check each guest's name against the list of \TextField[name=41,width=2.5cm,bordercolor=black,charsize=10pt]{\mbox{}} \TextField[name=42,width=2.5cm,bordercolor=black,charsize=10pt]{\mbox{}}, and follow the posted procedure.
            \item Members are responsible for making sure their guests follow the \TextField[name=51,width=2.5cm,bordercolor=black,charsize=10pt]{\mbox{}} and \TextField[name=52,width=2.5cm,bordercolor=black,charsize=10pt]{\mbox{}}.
            \item Bylaws and standing rules are located \TextField[name=61,width=2.5cm,bordercolor=black,charsize=10pt]{\mbox{}} \TextField[name=62,width=2.5cm,bordercolor=black,charsize=10pt]{\mbox{}} \TextField[name=63,width=2.5cm,bordercolor=black,charsize=10pt]{\mbox{}}.
            \item Any member who wishes to, may invite guests by throwing the \TextField[name=71,width=2.5cm,bordercolor=black,charsize=10pt]{\mbox{}}, located \TextField[name=72,width=2.5cm,bordercolor=black,charsize=10pt]{\mbox{}}.
            \item If you're the second-to-last person in the space, check that the remaining individuals are \TextField[name=81,width=2.5cm,bordercolor=black,charsize=10pt]{\mbox{}} (by asking to see their keyfob).
            \item When leaving, each member should \TextField[name=91,width=2.5cm,bordercolor=black,charsize=10pt]{\mbox{}} \TextField[name=92,width=2.5cm,bordercolor=black,charsize=10pt]{\mbox{}} after themselves and let the remaining members know they're leaving. (Also, figure out whether to leave the twitterbot on.)
            \item If you're the last one out, follow the \TextField[name=A1,width=2.5cm,bordercolor=black,charsize=10pt]{\mbox{}} located \TextField[name=A2,width=2.5cm,bordercolor=black,charsize=10pt]{\mbox{}}.
            \item Most tools in the space are owned by \TextField[name=B1,width=2.5cm,bordercolor=black,charsize=10pt]{\mbox{}}.
            \item In case of tool damage, or simply questions, post to \TextField[name=C1,width=2.5cm,bordercolor=black,charsize=10pt]{\mbox{}}.
            \item Information about tools may be found, or should be put, \TextField[name=D1,width=2.5cm,bordercolor=black,charsize=10pt]{\mbox{}} \TextField[name=D2,width=2.5cm,bordercolor=black,charsize=10pt]{\mbox{}} \TextField[name=D3,width=2.5cm,bordercolor=black,charsize=10pt]{\mbox{}}.
            \item Trash cans should be emptied by \TextField[name=E1,width=2.5cm,bordercolor=black,charsize=10pt]{\mbox{}}.
            \item New trash bags are kept in the \TextField[name=F1,width=2.5cm,bordercolor=black,charsize=10pt]{\mbox{}} \TextField[name=F2,width=2.5cm,bordercolor=black,charsize=10pt]{\mbox{}}.
            \item We recycle through SOCRRA. Recycling guidelines are on \TextField[name=G1,width=2.5cm,bordercolor=black,charsize=10pt]{\mbox{}}.
            \item The recycling bin goes out \TextField[name=H1,width=2.5cm,bordercolor=black,charsize=10pt]{\mbox{}} night.
            \item Sweeping, vacuuming, and mopping should be done by \TextField[name=I1,width=2.5cm,bordercolor=black,charsize=10pt]{\mbox{}}.
            \item Personal items brought into the space should be \TextField[name=J1,width=2.5cm,bordercolor=black,charsize=10pt]{\mbox{}}.
            \item Large objects/projects should have a \TextField[name=K1,width=2.5cm,bordercolor=black,charsize=10pt]{\mbox{}} completed for them.
            \item Member storage is restricted to one \TextField[name=L1,width=2.5cm,bordercolor=black,charsize=10pt]{\mbox{}} per dues-paying member.
            \item Items in member storage must be \TextField[name=M1,width=2.5cm,bordercolor=black,charsize=10pt]{\mbox{}} with name and contact info.
            \item Items in member storage must not protrude into the aisle, lest the \TextField[name=N1,width=2.5cm,bordercolor=black,charsize=10pt]{\mbox{}} hit them.
            \item Items in the space are available for everyone to use, unless they're in \TextField[name=O1,width=2.5cm,bordercolor=black,charsize=10pt]{\mbox{}} \TextField[name=O2,width=2.5cm,bordercolor=black,charsize=10pt]{\mbox{}} or have a completed \TextField[name=O3,width=2.5cm,bordercolor=black,charsize=10pt]{\mbox{}} \TextField[name=O4,width=2.5cm,bordercolor=black,charsize=10pt]{\mbox{}} attached.
            \item When cleaning up, if you don't know where something goes, contact the \TextField[name=P1,width=2.5cm,bordercolor=black,charsize=10pt]{\mbox{}} \TextField[name=P2,width=2.5cm,bordercolor=black,charsize=10pt]{\mbox{}} or post to \TextField[name=P3,width=2.5cm,bordercolor=black,charsize=10pt]{\mbox{}}.
            \item The front hallway is a \TextField[name=Q1,width=2.5cm,bordercolor=black,charsize=10pt]{\mbox{}} \TextField[name=Q2,width=2.5cm,bordercolor=black,charsize=10pt]{\mbox{}} and must always be kept clear.
        \end{enumerate}
    \end{Form}

\noindent\makebox[\linewidth]{\rule{\paperwidth}{0.4pt}}

    \begin{Form}
        \begin{itemize}
            \item[] Mentor name: \TextField[name=mname,width=4cm,bordercolor=black,charsize=10pt]{\mbox{}} Phone: \TextField[name=mphone,width=2cm,bordercolor=black,charsize=10pt]{\mbox{}}
            \item[] Mentor email: \TextField[name=memail,width=6cm,bordercolor=black,charsize=10pt]{\mbox{}}
            \item[] First followup date: \TextField[name=fdate,width=2cm,bordercolor=black,charsize=10pt]{\mbox{}} Second followup date: \TextField[name=sdate,width=2cm,bordercolor=black,charsize=10pt]{\mbox{}}
            \item[] New member name: \TextField[name=nname,width=4cm,bordercolor=black,charsize=10pt]{\mbox{}} Phone: \TextField[name=nphone,width=2cm,bordercolor=black,charsize=10pt]{\mbox{}}
            \item[] New member email: \TextField[name=nemail,width=6cm,bordercolor=black,charsize=10pt]{\mbox{}}
            \item[] \textbf{Mentor and new member: Add these dates to your calendar and touch base even if things are going well. Maybe take a picture of this section to remind yourself?}
        \end{itemize}
     \end{Form}


\normalsize
\newpage



\end{document}
